\subsection{Three-dimensional RTE solver \index{MYSTIC} (mystic)} 
\label{sec:mystic}

The Monte Carlo method is the most straightforward way to calculate 
(polarized) radiative transfer. In forward tracing mode  
individual photons are traced on their random paths through the atmosphere.
Starting from top of the atmosphere (for solar radiation), or being thermally
emitted by the atmosphere or surface, the photons are followed until they
hit the surface or leave again at top of the atmosphere (TOA). For
solar radiation, the start position is either a random location in the
TOA plane, with the direction determined by the solar zenith and
azimuth. Originally, the ``Monte Carlo for the physically correct tracing of
photons in cloudy atmospheres'' MYSTIC \citep{mayer2009} has been developed as
a forward tracing method for the calculation of irradiances and radiances in
\ifthreedmystic{3D} plane-parallel atmospheres. Later the model has been extended to fully
spherical geometry and a backward tracing mode \citep{emde2007}. The backward
photon tracing option speeds up the calculation of radiances and allows very
fast calculations in the thermal spectral range.  

\ifthreedmystic{MYSTIC handles three-dimensional water and ice clouds in a
one-dimensional background  atmosphere of molecular scatterers and
absorbers and aerosol particles.  MYSTIC also allows topography as well as 
inhomogeneous surface albedo and BRDF to be considered.}

MYSTIC is now a full vector code: It can handle polarization and
polarization-dependent scattering by randomly oriented particles,
i.e. clouds droplets and particles, aerosol particles, and molecules
\citep{emde2010}. To keep the computational time reasonable for
accurate calculations of e.g. polarized radiances in cloudy
atmospheres several ``tricks'' are required to speed up the
calculations. The first is the so called ``local estimate method''
\citep{marshak2005}. Using this method a photon contributes to the
final result of the calculation each time it is scattered. However, in
the presence of particles with strong forward scattering in the
simulated scene, such as clouds and large aerosols, the local estimate
method will produce so-called ``spikes'', these are rare photons whose
very large contribution to the result leads to slow convergence. The spike
problem can be resolved by using the ``Variance Reduction Optimal
Options Method'' \citep[VROOM,][]{buras2011a}, a collection of several
variance reduction methods which change the photon paths such that the
spikes disappear, but without altering the result (i.e.~the variance
reduction is ``unbiased'').

A detailed introduction to the Monte Carlo technique and in particular
to MYSTIC is given in \citet{mayer2009}. For specific questions 
concerning the Monte Carlo technique the reader is referred to the literature  
\citep{marchuk1980,collins1972,marshak2005,cahalan2005}.

MYSTIC is switched on by the option \code{rte\_solver mystic}. If no
other options are specified MYSTIC computes unpolarized quantities for
a \ifthreedmystic{3D} plane-parallel
atmosphere\ifthreedmystic{~(domain is divided into block-shaped 
boxes)}. If \codeidx{mc\_polarisation} is specified, polarized
quantities are computed. The option \codeidx{mc\_spherical 1D} enables
calculations in a 1D spherical model atmosphere. All MYSTIC-specific
options start with \code{mc\_} and
are described in detail in section~\ref{sec:uvspec_options}.

\ifthreedmystic{
MYSTIC may be provided with various 2D and 3D input files, in addition 
to the standard 1D input files of {\sl uvspec}. The input files may
include the following data:
\begin{itemize}
\item 3D water clouds
\item 3D ice clouds
\item 2D surface albedo
\item 2D elevation
\item 2D BRDF
%\item Lasers and detectors for lidar 
\end{itemize}

If none of these are specified, MYSTIC does basically a 1D calculation
and should compare well with other solvers, in particular \code{disort2}.

MYSTIC operates in a user-defined model-domain. Since periodic boundary 
conditions are applied, the user has to make sure that the quantity to be 
calculated is not affected by the boundaries. For maximum flexibility, 
the different grids (sample, cloud, elevation, albedo or BRDF) are 
completely independent. The only constraint is that the domain size
is equal for all grids used. E.g., clouds may be defined in 5x5x2~km$^3$ 
cubes while the surface albedo is defined on a 1x1~km$^2$ grid, 
the elevation on a 0.1x0.2~km$^2$ grid, and the output might 
be sampled on a 3x4~km$^2$ grid. The only restrictions are 
the available memory, the processor speed, and the requirements on 
the precision of the result (see section~\ref{sec:mc_memory}).

The sampling information for MYSTIC is defined in the input file (e.g.
with \codeidx{mc\_sample\_grid}, \codeidx{zout}, \codeidx{umu}, ...).
After initialization MYSTIC reports how it interpreted the input parameters 
which gives the user a chance to see if it really does what it is 
supposed to be doing.

\subsubsection{Coordinates}
Mystic in cartesian mode uses an ENU coordinate system to define the atmospheric state:
\begin{itemize}
    \item{\emph{x}: east}
    \item{\emph{y}: north}
    \item{\emph{z}: up}
\end{itemize}

\begin{figure}[h!]
\centering
\tdplotsetmaincoords{70}{20}
\begin{tikzpicture}[scale=5,tdplot_main_coords]

    \coordinate (O) at (0,0,0);
    \coordinate (Ex) at (1,0,0);
    \coordinate (Ey) at (0,1,0);
    \coordinate (Ez) at (0,0,1);


    \draw[thick, ->] (-0.2,0,0) -- (Ex) node[anchor=west]{$x~\textrm{(east)}$};
    \draw[thick, ->] (0,-0.2,0) -- (Ey) node[anchor=west]{$y~\textrm{(north)}$};
    \draw[thick, ->] (0,0,0) -- (Ez) node[anchor=south]{$z~\textrm{(up)}$};

    \tdplotsetcoord{P}{1}{45.57}{60};
    \draw[thick, ->, color=red] (O) -- (P) node[anchor=south west]{sensor};
    \draw[dashed, color=red] (O) -- (Pxy);
    \draw[dashed, color=red] (P) -- (Pxy) node[midway,anchor=west]{$\textrm{umu} = 0.7$};

    \tdplotdrawarc[color=red]{(O)}{0.3}{90}{60}{xshift=0.3cm,anchor=north west}{$\textrm{phi} = 30\degree$};
\end{tikzpicture}
\caption{Sensor coordinate specification (\codeidx{umu} and \codeidx{phi}). The red arrow is pointing into the sensor, thus a sensor with positive umu values is looking downwards.}
\label{fig_sensor_coordinates}
\end{figure}
Generally, the sensor orientation is specified using the \codeidx{umu} and \codeidx{phi} input options as shown in figure~\ref{fig_sensor_coordinates}.
The location of the sensor is given in the same $x$, $y$, $z$ coordinate system by use of the option \codeidx{mc\_sensorposition}.
The position of the sun is specified either using \codeidx{latitude}, \codeidx{lognitude} and \codeidx{time} or using \codeidx{sza} and \codeidx{phi0}.
In the latter case, the angles are (maybe depending on the mindset of the reader), defined slightly different as shown in figure~\ref{fig_sun_coordinates}.

\begin{figure}[h!]
\centering
\tdplotsetmaincoords{70}{20}
\begin{tikzpicture}[scale=5,tdplot_main_coords]

    \coordinate (O) at (0,0,0);
    \coordinate (Ex) at (1,0,0);
    \coordinate (Ey) at (0,1,0);
    \coordinate (Ez) at (0,0,1);


    \draw[thick, ->] (-0.2,0,0) -- (Ex) node[anchor=west]{$x~\textrm{(east)}$};
    \draw[thick, ->] (0,-0.2,0) -- (Ey) node[anchor=west]{$y~\textrm{(north)}$};
    \draw[thick, ->] (0,0,0) -- (Ez) node[anchor=south]{$z~\textrm{(up)}$};

    \tdplotsetcoord{P}{1}{30}{210};
    \draw[thick, ->, color=orange] (O) -- (P) node[anchor=south east]{sun};
    \draw[dashed, color=orange] (O) -- (Pxy);
    \draw[dashed, color=orange] (P) -- (Pxy);

	\tdplotsetthetaplanecoords{210};
    \tdplotdrawarc[tdplot_rotated_coords, color=orange]{(O)}{0.5}{0}{30}{xshift=0.7cm,anchor=west}{$\textrm{sza} = 30\degree$}

    \tdplotdrawarc[color=orange]{(O)}{0.15}{270}{210}{yshift=-0.2cm,anchor=north}{$\textrm{phi0} = 60\degree$};
\end{tikzpicture}
\caption{Sun coordinate specification (sza and phi0). The orange arrow is pointing into the sun.}
\label{fig_sun_coordinates}
\end{figure}



\begin{figure}[h!]
\centering
\tdplotsetmaincoords{70}{20}
\pgfmathsetmacro{\thetavec}{20}
\pgfmathsetmacro{\phivec}{320}
\begin{tikzpicture}[scale=5,tdplot_main_coords]

    \coordinate (O) at (0,0,0);
    \coordinate (Ex) at (1,0,0);
    \coordinate (Ey) at (0,1,0);
    \coordinate (Ez) at (0,0,1);


    \draw[thick, ->] (-0.2,0,0) -- (Ex) node[anchor=west]{$x~\textrm{(east)}$};
    \draw[thick, ->] (0,-0.7,0) -- (Ey) node[anchor=west]{$y~\textrm{(north)}$};
    \draw[thick, ->] (0,0,-0.7) node[anchor=north]{nadir} -- (Ez) node[anchor=south]{$z~\textrm{(up, zenith)}$};

    \tdplotsetcoord{P1}{0.6}{60}{160}
    \tdplotsetcoord{P2}{0.6}{60}{100}
    \tdplotsetcoord{P3}{0.6}{100}{100}
    \tdplotsetcoord{P4}{0.6}{100}{160}
    \tdplotsetcoord{P3a}{0.6}{90}{100}
    \tdplotsetcoord{P3b}{0.7}{90}{100}

	\draw[dashed,color=orange] (O) -- (P1) node[anchor=east]{$(0,N_y-1)$};
	\draw[dashed,color=orange] (O) -- (P2) node[xshift=0.5cm,anchor=west]{$(N_x-1,N_y-1)$};
	\draw[dashed,color=orange] (O) -- (P3) node[xshift=0.5cm,anchor=west]{$(N_x-1,0)$};
	\draw[dashed,color=orange] (O) -- (P4) node[xshift=-1.2cm,anchor=north east]{$(0,0)$};
	\draw[dashed,color=black] (P3a) -- (P3b);

    \tdplotdrawarc[color=black,dotted]{(0,0,0)}{0.6}{160}{-90}{}{}
    \tdplotdrawarc[color=orange]{(P1z)}{0.519}{100}{160}{}{}
    \tdplotdrawarc[dashed,color=blue]{(P1z)}{0.519}{95}{100}{}{}
    \tdplotdrawarc[color=orange]{(P3z)}{0.59088}{100}{160}{}{}

    \tdplotdrawarc[color=black,dashed]{(0,0,0)}{0.6}{270}{160}{xshift=0.1cm,anchor=west}{$\phi_1 = 110\degree$}
    \tdplotdrawarc[color=black,dashed]{(0,0,0)}{0.7}{270}{100}{anchor=east}{$\phi_2 = 170\degree$}
    \tdplotsetthetaplanecoords{100}
    \tdplotdrawarc[tdplot_rotated_coords,color=orange]{(0,0,0)}{0.6}{60}{100}{}{}
    %\tdplotdrawarc[tdplot_rotated_coords,dotted,color=black]{(0,0,0)}{0.6}{0}{360}{}{}
    %\tdplotdrawarc[tdplot_rotated_coords,dashed,color=black]{(0,0,0)}{0.6}{100}{90}{}{}
    \tdplotsetthetaplanecoords{155}
    \tdplotdrawarc[tdplot_rotated_coords,dashed,color=blue]{(0,0,0)}{0.6}{100}{180}{anchor=east}{$\theta_1 = 80\degree$}
    \tdplotsetthetaplanecoords{95}
    \tdplotdrawarc[tdplot_rotated_coords,dashed,color=blue]{(0,0,0)}{0.6}{60}{180}{anchor=west}{$\theta_2 = 120\degree$}
    \tdplotsetthetaplanecoords{160}
    \tdplotdrawarc[tdplot_rotated_coords,color=orange]{(0,0,0)}{0.6}{60}{100}{}{}
    %\tdplotdrawarc[tdplot_rotated_coords,dotted,color=black]{(0,0,0)}{0.6}{0}{360}{}{}
    %\tdplotdrawarc[tdplot_rotated_coords,dashed,color=black]{(0,0,0)}{0.6}{80}{90}{}{}

\end{tikzpicture}
    \caption{Panorama coordinates for the option \codeidx{mc\_panorama\_alignment} \code{zenith} (default). The camera is located at the coordinate system origin. The image is recorded on a spherical shell inside the orange box, which is behind the origin. The boundaries of this box are given by the parameters $\phi_1$, $\phi_2$, $\theta_1$, $\theta_2$ to \codeidx{mc\_panorama\_view}. $\phi_1$ and $\phi_2$ are measured in the $x-y$-plane, beginning in the south. $\theta_1$ and $\theta_2$ measure the "latitude" where nadir is 0. The coordinates given in orange show the $(x, y)$ coordinates of the resulting image. The $(x, y)$ coordinates thus correspond to $(\theta, \phi)$ coordinates.}
\label{fig_panorama_coords_zenith}
\end{figure}


\begin{figure}[h!]
\centering
\tdplotsetmaincoords{70}{20}
\pgfmathsetmacro{\thetavec}{36.87}
\pgfmathsetmacro{\phivec}{-50}
\begin{tikzpicture}[scale=5,tdplot_main_coords]

    \coordinate (O) at (0,0,0);
    \coordinate (Ex) at (1,0,0);
    \coordinate (Ey) at (0,1,0);
    \coordinate (Ez) at (0,0,1);


    \draw[thick, ->] (-0.1,0,0) -- (Ex) node[anchor=west]{$x~\textrm{(east)}$};
    \draw[thick, ->] (0,-0.6,0) -- (Ey) node[anchor=west]{$y~\textrm{(north)}$};
    \draw[thick, ->] (0,0,0) -- (Ez) node[anchor=south]{$z~\textrm{(up)}$};

    \tdplotsetcoord{P}{0.6}{\thetavec}{\phivec}
    \tdplotsetcoord{Q}{0.6}{53.13}{130}
    \draw[dashed, color=blue] (P) -- (Pxy) node[near start,anchor=west]{$\textrm{umu} = -0.8$};
    \draw[dashed, color=blue] (O) -- (Pxy);
    \tdplotdrawarc[color=red]{(O)}{0.48}{270}{130}{xshift=0.3cm,anchor=south east}{$\textrm{phi} = 140\degree$};
    \draw[dashed, color=red] (Q) -- (Qxy);

    \tdplotsetrotatedcoords{\phivec}{\thetavec}{0}

    \draw[tdplot_rotated_coords, thick, ->, color=magenta] (0,-0.1,0) -- (0,0.6,0) node[anchor=south east]{$x'$};
    \draw[tdplot_rotated_coords, thick, ->, color=red] (0.4,0,0) -- (-0.6,0,0) node[anchor=north west]{$y'$};
    \draw[tdplot_rotated_coords, thick, -, color=red] (0.6,0,0) -- (0.53,0,0);
    \draw[tdplot_rotated_coords, thick, ->, color=blue] (0,0,-0.6) -- (0,0,0.6) node[anchor=south east]{$z'$};

    \tdplotdrawarc[tdplot_rotated_coords,color=black,dotted]{(0,0,0)}{0.6}{0}{345}{}{};
    \tdplotdrawarc[tdplot_rotated_coords,color=black,dashed]{(0,0,0)}{0.6}{0}{-15}{anchor=north west}{$\phi_1=15\degree$};
    \tdplotdrawarc[tdplot_rotated_coords,color=black,dashed]{(0,0,0)}{0.8}{0}{-45}{anchor=north west}{$\phi_2=45\degree$};

    \tdplotdrawarc[tdplot_rotated_coords,color=orange]{(0,0,0.1042)}{0.59088}{-15}{-45}{}{}
    \tdplotdrawarc[tdplot_rotated_coords,color=orange]{(0,0,-0.1042)}{0.59088}{-15}{-45}{}{}

    \tdplotsetrotatedthetaplanecoords{0}
    \tdplotdrawarc[tdplot_rotated_coords,color=black,dotted]{(0,0,0)}{0.6}{0}{360}{}{}
    \tdplotsetrotatedthetaplanecoords{-15} %note that this changes the rotated frame already
    \tdplotdrawarc[tdplot_rotated_coords,color=orange]{(0,0,0)}{0.6}{80}{100}{}{}
    \tdplotdrawarc[tdplot_rotated_coords,dashed,color=blue]{(0,0,0)}{0.6}{180}{100}{yshift=1cm,anchor=west}{$\theta_1=80\degree$}
    \tdplotsetrotatedthetaplanecoords{-30}
    \tdplotdrawarc[tdplot_rotated_coords,color=orange]{(0,0,0)}{0.6}{80}{100}{}{}
    \tdplotsetrotatedthetaplanecoords{-5}
    \tdplotdrawarc[tdplot_rotated_coords,dashed,color=blue]{(0,0,0)}{0.6}{180}{80}{yshift=0.5cm,anchor=east}{$\theta_2=100\degree$}

\end{tikzpicture}
    \caption{Panorama corrdinates for \codeidx{mc\_panorama\_alignment} \code{mu}. In \emph{mu} aligned mode, the panorama is taken as in \emph{zenith}-mode, however the whole coordinate system is rotated using the parameters \codeidx{umu} and \codeidx{phi} yielding the new coordinates $x'$, $y'$, $z'$. If $\textrm{umu} = -1$ and $\textrm{phi} = 180\degree$, the coordinates are exactly the same as in \emph{zenith} mode. Be aware that if \codeidx{umu} is $1$ or $-1$, the value of \codeidx{phi} is not accepted and the system behaves as if $\textrm{phi} = 180\degree$. In this case, the equivalent rotation can be achieved by $\phi_1$ and $\phi_2$.}
\label{fig_panorama_coords_mu}
\end{figure}

If panoramic images are calculated, the sensor specification is extended by a definition for the field of view of the camera an the option \codeidx{mc\_panorama\_view}.
The behavior of this specification can be changed by the \codeidx{mc\_panorama\_alignment}.
In the default mode (\codeidx{mc\_panorama\_alignment} \code{zenith}), \code{umu} and \code{phi} are ineffective.
The view specificatiion behaves as described in figure~\ref{fig_panorama_coords_zenith}.
If \codeidx{mc\_panorama\_alignment} \code{mu} is activated, the user can basically rotate the camera.
The angles are defined as in figure~\ref{fig_panorama_coords_mu}.

\subsubsection{Three-dimensional clouds}

The options \codeidx{wc\_file 3D}
and \codeidx{ic\_file 3D} may be used to define 3D water and ice
cloud properties respectively. The model atmosphere may contain both 3D 
(defined in a \code{wc\_file/ic\_file 3D}) and 1D clouds (defined in a 
\code{wc\_file/ic\_file 1D}), but for obvious reasons not in the same layer. The format
of the files is explained in section~\ref{sec:uvspec_options}. 

The conversion from microphysical to optical properties 
is done identically for 1D and 3D clouds and is defined 
by the options \codeidx{wc\_properties} and \codeidx{ic\_properties}. All
possible settings for \code{wc\_properties} and \code{ic\_properties}
have been implemented in MYSTIC (e.g., hu, mie, yang, HEY, baum ...).
Clouds can be defined eiter as ASCII or as netCDF files.

\paragraph{ASCII}
The following example shows the easiest possible 3D 
cloud file which defines a checkerboard grid of clouds:
\begin{Verbatim}[fontsize=\footnotesize, frame=single, samepage=true]
  2 2 5 3
  1.0 1.0 0 1 2 3 4 5 

  1 1 2 0.1 10
  2 2 2 0.1 10
\end{Verbatim}
For the layer between 1 and 2 km a 3D cloud is defined with liquid water
content 0.1 $g/m^3$ and effective radius 10 $\mu$m for horizontal 
boxes (1,1) and (2,2). Boxes (1,2) and (2,1) are cloudless. The result are
cubic clouds with a volume of 1x1x1~km$^3$. The microphysical properties are
converted to optical properties according to \code{wc\_properties}.  

\paragraph{netCDF}
If the provided cloud file is a netCDF instead of an ASCII file, this will be
detected automatically. The netCDF file must have the following structure:
\begin{Verbatim}[fontsize=\footnotesize, frame=single, samepage=true]
netcdf cloud {
dimensions:
	ny = 2 ;
	nx = 2 ;
	nz = 1 ;
	nz_lev = 2 ;
variables:
	double lwc(ny, nx, nz) ;
	double z(nz_lev) ;
	double reff(ny, nx, nz) ;

// global attributes:
	:cldproperties = 3 ;
	:dx = 1. ;
	:dy = 1. ;
}
\end{Verbatim}
The order of the dimensions is important, and the size of \code{nz\_lev} must be
exactly \code{nz + 1}. The \code{cldproperties} attribute is equal to the flag
in the ASCII format. The variable names change for different \code{cldproperties}
according to the following table:
\begin{itemize}
    \item{1: \quad \code{ext g1 ssa}}
    \item{2: \quad \code{ext reff}}
    \item{3: \quad \code{lwc reff}}
\end{itemize}

\subsubsection{Two-dimensional surface albedo}

A 2D surface albedo may be defined using the option
\codeidx{mc\_albedo\_file}. The format of the file is explained in
section~\ref{sec:uvspec_options}. 

\subsubsection{Topography}

A 2D elevation may be defined using the option 
\codeidx{mc\_elevation\_file}. The ASCII format is explained in
section~\ref{sec:uvspec_options}. For the netcdf format please refer
to \file{README.MC}. 

Caution:
\begin{itemize}
\item To avoid confusion, do not specify an \code{altitude} different
  from 0 in the uvspec input file.
\item There may be problems if the 2D surface hits 0 
  (more testing required). Use 0.001 as lowest altitude.
\item  The elevation file MUST BE PERIODIC
\end{itemize}

Between the grid points, the surface is interpolated 
bilinearely: 
\begin{equation}
  z(x,y) = a + bx + cy + dxy
\end{equation}
where $z$ is the altitude and the coefficients $a$, $b$, $c$, and $d$
are determined from 
the altitudes specified at the four corners of each pixel.
Slant surfaces are treated as they should be; e.g., photons
may be reflected downward at snow-covered mountain sides.

\subsubsection{Bidirectional reflectance distribution function}
MYSTIC allows different homogenous or 2D-inhomogeneous BRDFs. The 
implementation of the different BRDFs is rather heterogeneous,   
as were the requirements when each of them was first implemented.
Please note that BRDFs currently do not work together with topography. The 
very simple reason for that is in structured terrain reflection 
polar angles larger than 90 degree are possible for which the 
existing parameterizations do not provide data. The following  
parameterizations are included:
\begin{itemize}
\item Water BRDF by \citet{cox54a,cox54b}; currently only homogeneous; switched 
  on with \codeidx{brdf\_cam u10}, ... like in the 1D case
\item AMBRALS (Algorithm for Modeling[MODIS] Bidirectional Reflectance
  Anisotropies of the Land Surface; \citet{wanner97}), see also
  \url{http://i3rc.gsfc.nasa.gov/phase3/brasil/phase3_step1_brasil.htm}.
  AMBRALS is three-parameter BRDF fit for vegetated and non-vegetated surfaces.
  A \codeidx{mc\_ambrals\_file} may be specified which contains the 
  3 parameters for each surface pixel. The format is like 
  the format of 2D surface albedo file, except that each data line 
  contains (iso, vol, geo) instead of only one albedo value.
  Obviously, wavelength-dependent AMBRALS BRDFs are not 
  possible at present.
\item RPV \citep{rahman93a}, a three parameter 
  fit for vegetated and non-vegetated surfaces. RPV is currently the 
  most flexible surface description in MYSTIC as it is currently 
  the only way to define a wavelength-dependent 2D BRDF. 
  Two different ways exist to define an RPV BRDF:
  \begin{itemize}
  \item 1D, using the options \codeidx{brdf\_rpv k} ... like in the 1D case
  \item using \codeidx{mc\_rpv\_file} and \codeidx{mc\_rpv\_type}.
    The first is a RPV
    type for each surface pixel; type is simply a label like 
    "Grass" or "1" or whatever. For each of these types,
    \code{mc\_rpv\_type} must have an entry which connects the type to
    a file which contains wavelength-dependent RPVs. Sounds
    complicated but is a very efficient way to define a
    wavelength-dependent 2D RPV. Label ``CaM'' is reserved for Cox and Munk 
    and invokes the Cox and Munk ocean BRDF for the respective pixel. Thus
    it is possible to combine land and ocean BRDFs.
  \end{itemize}
\item For calculations including polarization the option
  \codeidx{bpdf\_tsang\_u10} is available which calculates polarized
  bidirectional reflection from water surfaces. 
\end{itemize}
}

\subsubsection{MYSTIC output}

{\sl uvspec} will print its output (horizontally averaged 
irradiance and actinic flux) usually to stdout. MYSTIC provides 
several additional output 
files. We have to distinguish two classes of output: 
Monochromatic and spectral output where the latter can be recognized 
by the extension ``.spc''. Monochromatic output files

\begin{itemize}
\item mc.flx - irradiance, actinic flux at levels
\item mc.rad - radiance 
\item mc.abs - absorption/emission
\item mc.act - actinic flux, averaged over layers
\end{itemize}  

are generated only for the case of a calculation where MYSTIC is called 
only once. That is, a monochromatic calculation without subbands introduced 
by \code{mol\_abs\_param}. They contain ``plain'' MYSTIC output, without
consideration of extraterrestrial irradiance, sun-earth-distance, spectral
integration, etc. As such they are mainly interesting for MYSTIC developers
or for users interested in artificial cases and photon statistics since 
they are as close as possible to the photon statistics of MYSTIC: e.g. 
the ``irradiance'' in these files is basically the number of photons arriving
at the detector divided by the number of photons traced. In addition to the 
average, a standard deviation of the result can be calculated online which is 
stored in ``.std''. 

For most real-world applications the user will prefer the ``.spc'' files

\begin{itemize}
 \item mc.flx.spc - spectral irradiance, actinic flux at levels
 \item mc.rad.spc - spectral radiance at levels
 \item ...
\end{itemize}

In contrast to the monochromatic files which are transmittances (E/E0, L/E0,
...) the data in ".spc" is "fully calibrated" output, as for all other
solvers. "fully calibrated" means multiplied with the extraterrestrial 
irradiance, corrected for the Sun-Earth distance, integrated/summed
over wavelength, etc. Please be aware that such a calculation might require
a lot of memory because output is stored as a function of x, y, z, and
wavelength (and possibly polarization, if you switched on \code{mc\_polarisation}).
E.g. a comparatively harmless "mol\_abs\_param kato2" calculation with an sample grid of 100x100 pixels
at 10 altitudes would imply about 100x100x10x148 = 14,800,000 (Nx$\cdot$Ny$\cdot$Nz$\cdot$Nlambda) grid points. Depending 
on the output chosen (irradiance, radiance, ...) up to six floating 
point numbers need to be stored which amounts to 360 MBytes. Depending 
on the post-processing in {\sl uvspec}, this memory may actually be used 
twice which then would be 720 MBytes. 

\begin{description}
\item[mc.flx / mcNN.flx] 
  The output file \code{mc.flx} contains the irradiance
  at the surface defined by elevation file. Note that
  this output is {\bf not} for $z=0$, but for the actual 2D surface:
  \begin{Verbatim}[fontsize=\footnotesize, frame=single, samepage=true]
    500  500 0 0.325889 0 0 0.441766 0
    500 1500 0 0.191699 0 0 0.267122 0
    500 2500 0 0.210872 0 0 0.420268 0
  \end{Verbatim}
  The columns are:
  \begin{enumerate}
  \item  x [m] (pixel center)
  \item  y [m] (pixel center)
  \item  direct transmittance
  \item  diffuse downward transmittance
  \item  diffuse upward transmittance
  \item  direct actinic flux transmittance
  \item  diffuse downward actinic flux transmittance
  \item  diffuse upward actinic flux transmittance
  \end{enumerate}
  The transmittance is defined as irradiance devided by the extraterrestrial irradiance.
  It is not corrected for Sun-Earth-Distance.

  Note that even for an empty atmosphere, the transmittance
  would not be 1 but cos(SZA), due to the slant incidence of the radiation. 
  
  The output files \code{mcNN.flx} contain the irradiances
  at different model levels - one for each \code{zout}. \code{NN} is the
  number of the output level counted from the bottom (ATTENTION: Levels are
  counted from 0 here!). The file format is the same as in \code{mc.flx}.

  (If interested in surface quantities, please use the irradiance data at the
  surface from \code{mc.flx}, not from \code{mc00.flx}; the data from 
  \code{mc00.flx} or whatever layer coincides with the 
  surface may be wrong for technical reasons).

\item[mc.rad / mcNN.rad] 
  The output file \code{mc.rad} contains the radiance 
  at the surface defined by \code{elevation\_file}. Note that
  this output is {\bf not} for $z=0$, but for the actual 2D surface:
  \begin{Verbatim}[fontsize=\footnotesize, frame=single, samepage=true]
    500  500 45 270 0.0239094 0 0.0623305 0.063324
    500 1500 45 270 0.0239094 0 0.0602891 0.063156
  \end{Verbatim}
  The columns are:
  \begin{enumerate}
  \item  x [m] (pixel center)
  \item  y [m] (pixel center)
  \item  viewing zenith angle  [deg]
  \item  viewing azimuth angle [deg]
  \item  aperture solid angle [sterad]
  \item  direct  radiance component
  \item  diffuse radiance component
  \item "escape" radiance 
  \end{enumerate}
  For almost all applications you may safely ignore the ``direct'' and
  ``diffuse'' radiance components and use only the escape radiance. If the 
  latter is 0 then you probably forgot to switch on \code{mc\_escape}.
  The "escape" radiance is the radiance "measured" by an ideal
  instrument with 0$^\circ$ opening angle. It is only calculated 
  when \codeidx{mc\_escape} is selected and it usually converges 
  much faster than the "cone sampled" radiance in column 7. 
  It is recommended to always use \code{mc\_escape} for radiance
  calculations. For the ``direct'' and ``diffuse'' radiance, photons
  falling into the aperture are counted. This might be an option for 
  instruments with a very large aperture only because otherwise the result
  is noisy. 

  The output files \code{mcNN.rad} contain the radiances
  at different model levels - one for each \code{zout}. \code{NN} is the
  number of the output level counted from the bottom (ATTENTION: Levels are
  counted from 0 here!). The file format is the same as in \code{mc.rad}.
  
  (If interested in surface quantities, please use the radiance data
  at the surface from \code{mc.rad}, not from \code{mc00.rad}; the data from 
  \code{mc00.rad} or whatever layer coincides with the 
  surface may be wrong for technical reasons).

\item[mcNN.abs] 
  The file \code{mcNN.abs} includes the absorption per unit area in the given layer. 
  NN is the number of the output layer on the atmospheric 
  grid (counted from the bottom, starting from 1). This file is generated if \code{mc\_forward\_ouput absorption} 
  or \code{mc\_forward\_output emission} is specified.
  
  The columns are:
  \begin{enumerate}
  \item  x [m] (pixel center)
  \item  y [m] (pixel center)
  \item  absorption/emission/heating rate (W/m$^2$)
  \end{enumerate} 
  
  If multiplied by the extraterrestrial irradiance, the column 
  absorption in W/m$^2$ is obtained. In a 1D atmosphere, with a solar source,
  absorption = e\_net(top) - e\_net(bottom) (this is not true for a thermal
  source because then emission needs to be considered; see below). If
  \code{mc\_forward\_output emission} is specified, the file contains the thermal emission of
  the layer per unit area, that is, the Planck function times the optical
  thickness of the layer times 4$\pi$ (angular integral of the Planck
  radiance). If \code{mc\_forward\_output heating} is specified, the heating rate per unit area is  
  provided instead of absorption (in the same units as absorption).
  For a solar source, the heating rate is identical to the absorption. 
  In the thermal, however, each emitted photon is counted as cooling 
  and hence the heating rate may be negative. In a 1D atmosphere, with a
  solar or thermal source, absorption = e\_net(top) - e\_net(bottom). 

  For computational efficiency reasons \code{mcNN.abs} is not provided on the
  sample grid but on the atmospheric grid. For the same reason, results are 
  only calculated for 3D layers. In order to obtain 3D absorption for 1D
  cloudless layer, you need to specify an optically very thin 3D cloud, e.g.
  LWC/IWC = 10$^{-20}$ g/m$^3$ (yes, this is a dirty trick but a necessary one). 

\item[mcNN.act] 
  \code{mcNN.act} contains the 4$\pi$ actinic flux in the given layer,
  calculated from the absorbed energy (per unit area) divided by the 
  absorption optical thickness of the layer. In contrast to the 
  actinic flux in \code{mcNN.flx}, this is a layer quantity the 
  accuracy of which is generally much better than the level quantities 
  which are calculated from radiance / cos (theta). As for the absorption
  above, \code{mcNN.act} is not provided on the sample grid but on the
  atmospheric grid. NN is the number of the output layer (counted from the
  bottom, starting from 1). This file is generated if \codeidx{mc\_forward\_output actinic} is specified. 
    
  The columns are:
  \begin{enumerate}
  \item  x [m]
  \item  y [m]
  \item  actinic flux (W/m$^2$)
  \end{enumerate}

\end{description}  

The spectral files are as follows:
\begin{description}
 \item[mc.flx.spc]
   \ifthreedmystic{This is the most versatile and useful irradiance /
     actinic flux output of MYSTIC}:
  \begin{Verbatim}[fontsize=\footnotesize, frame=single, samepage=true]
  400.0 0 0 0 1.0e+00 0.0e+00 1.5067e-01 1.0e+00 0.0e+00 3.5044e-01
  401.0 0 0 0 1.0e+00 0.0e+00 1.5044e-01 1.0e+00 0.0e+00 3.5863e-01
  402.0 0 0 0 1.0e+00 0.0e+00 1.5022e-01 1.0e+00 0.0e+00 3.4755e-01
  \end{Verbatim}

  The columns are:
  \begin{enumerate}
  \item  wavelength [nm]
  \item  ix (0 ... Nx-1)
  \item  iy (0 ... Ny-1)
  \item  iz (0 ... Nz-1)
  \item  direct irradiance
  \item  diffuse downward irradiance
  \item  diffuse upward irradiance
  \item  direct actinic flux
  \item  diffuse downward actinic flux
  \item  diffuse upward actinic flux
  \end{enumerate}
  These numbers are created the same way as the standard {\sl uvspec} output. That
  is, they are multiplied with the extraterrestrial irradiance, corrected for
  Sun-Earth-distance, integrated over wavelength, converted to reflectivity or 
  brightness temperature, etc. 

 \item[mc.rad.spc]
   \ifthreedmystic{This is the most versatile and useful radiance output of
   MYSTIC}:
  \begin{Verbatim}[fontsize=\footnotesize, frame=single, samepage=true]
  400.0 0 0 0 0.0398276
  401.0 0 0 0 0.0396459
  402.0 0 0 0 0.0398005
  \end{Verbatim}

  The columns are:
  \begin{enumerate}
  \item  wavelength [nm]
  \item  ix (0 ... Nx-1)
  \item  iy (0 ... Ny-1)
  \item  iz (0 ... Nz-1)
  \item  radiance 
  \end{enumerate}
  These numbers are created the same way as the standard {\sl uvspec}
  output. That is, they are multiplied with the extraterrestrial irradiance,
  corrected for Sun-Earth-distance, integrated over wavelength, converted to
  reflectivity or brightness temperature, etc. 

  If the {\bf polarized} mystic is used (\code{mc\_polarisation}) then the four
  components of the Stokes vector (I,Q,U,V) are output for each wavelength and
  grid point, in four separate lines. 

\ifthreedmystic{
 \item[mc.abs.spc]
  \code{mc.abs.spc} contains the absorption per unit area for all layers. This
  file is generated if \code{mc\_forward\_output absorption} or \code{mc\_forward\_output emission} is specified.
  
  The columns are:
  \begin{enumerate}
  \item  wavelength [nm]
  \item  ix (0 ... Nx-1)
  \item  iy (0 ... Ny-1)
  \item  iz (0 ... Nz-1)
  \item  absorption/emission/heating rate
  \end{enumerate} 
  
  These numbers are created the same way as the standard {\sl uvspec} output. That
  is, they are multiplied with the extraterrestrial irradiance, corrected for
  Sun-Earth-distance, integrated over wavelength, converted to reflectivity or 
  brightness temperature, etc.  The unit is determined by an extra option to 
  \code{mc\_forward\_output absorption} or \code{mc\_forward\_output emission}. 

  For computational efficiency reasons \code{mc.abs.spc} is not provided on the
  sample grid but on the atmospheric grid. For the same reason, results are 
  only calculated for 3D layers. In order to obtain 3D absorption for 1D
  cloudless layer, you need to specify an optically very thin 3D cloud, e.g.
  LWC/IWC = 10$^{-20}$ g/m$^3$ (yes, this is a dirty trick but a necessary one). 

 \item[mc.act.spc]
  \code{mcNN.act} contains the 4$\pi$ actinic flux in the given layer,
  calculated from the absorbed energy (per unit area) divided by the 
  absorption optical thickness of the layer. In contrast to the 
  actinic flux in \code{mc.act.flx}, this is a layer quantity the 
  accuracy of which is generally much better than the level quantities 
  which are calculated from radiance / cos (theta). As for the absorption
  above, \code{mcNN.act} is not provided on the sample grid but on the
  atmospheric grid. NN is the number of the output layer (counted from the
  bottom, starting from 1). This file is generated if \codeidx{mc\_forward\_output actinic} is specified. 
    
  The columns are:
  \begin{enumerate}
  \item  wavelength [nm]
  \item  ix (0 ... Nx-1)
  \item  iy (0 ... Ny-1)
  \item  iz (0 ... Nz-1)
  \item  actinic flux
  \end{enumerate}
}
\end{description}

\ifthreedmystic{
\subsubsection{Memory requirements and computational speed}
\label{sec:mc_memory}

\begin{enumerate}
\item  3D clouds are defined on a 3D grid and the amount of 
  memory required by MYSTIC is usually determined by 
  the 3D cloud grid. Several variables need to be stored 
  for each grid cell, including the optical properties of 
  the cell, the grid cell absorption, etc. Umpteen (20-100) 
  bytes are typically required for each grid cell. 
  Computational time is to a large degree determined by 
  the total optical thickness of the cloud because higher 
  extinction means more scattering events and a longer 
  photon path. To a lesser degree, the computational time 
  is influenced by the number of cells but this may become 
  important if the radiance is calculated using 
  local estimates.

\item 2D albedo requires some memory but usually less than 
  clouds because it is defined on a 2D grid compared to the 
  3D cloud grid. Only one double (8 bytes on a 32 bit machine) 
  is stored per pixel. A high resolution has no influence 
  on computational time.

\item 2D elevation requires somewhat more memory than a 2D albedo 
  because 5 doubles are stored per pixel. Higher resolution 
  will lead to higher computational times. 

\item The sample grid by itself has only little influence on 
  computational time because for a given number of photons, 
  the computational time does not depend on the number of 
  grid cells. The precision of the result, however depends 
  strongly on the number of grid cells Nx $\cdot$ Ny, as it 
  decreases with sqrt(Nx$\cdot$Ny). The number of altitude 
  levels has only little influence on the computational 
  time but of course a large influence on the memory 
  requirements. The calculation of radiances has a large 
  impact on computational time if local estimates are used.
  % CE: In current version it is only possible to compute 1 direction at a time
  %In the worst case, the computational time may 
  %scale directly with the number of directions for which the 
  %radiance is to be calculated.
\end{enumerate}
}  
