\chapter{Calculation of optical properties - \code{mie}}
\label{sec:optprop}

\index{Mie}

{\sl libRadtran} includes the tool \code{mie} to calculate optical
properties of spherical particles. Two different efficient and well
tested Mie codes are implemented: The
one by \citet{wiscombe80a} and the one by \citet{bohren1998}. Scattering phase matrices and
corresponding Legendre polynomials can currently only be calculated
using the code by \citet{wiscombe80a}.

\section{Basic usage}

\subsection{Running \code{mie}}

Mie scattering calculations are performed for a specified
wavelength interval. The \code{mie} program reads input from standard input, 
and outputs to standard output or to a file. If
\code{output\_user netcdf} is specified \code{mie} generates a file
that can be used for radiative transfer calculations with
\code{uvspec}.

The \code{mie} tool is normally invoked in the following
way: 

\begin{Verbatim}[fontsize=\footnotesize]   
  mie < input_file > output_file 
\end{Verbatim}

\strong{Warning:} Please note the error checking on input variables 
is very scarce at the moment. Hence, if you provide erroneous input, 
the outcome is unpredictable.

\subsection{The \code{mie} input file}

The \code{mie} input file consists of single line entries, each making
up a complete input to the mie program. First on the line comes
the parameter name, followed by one or more parameter values. The 
parameter name and the parameter values are seperated by white space.

Filenames are entered without any surrounding single or double quotes.

Comments are introduced by a \code{\#}. Blank lines are ignored.

\subsection{Model output}

The standard output (stdout) of the mie program is one line for each 
wavelength and each effective radius.
The format of the output line is
\begin{Verbatim}[fontsize=\footnotesize, frame=single, samepage=true]   
  lambda refrac_real refrac_imag qext omega gg spike pmom
\end{Verbatim}
The keywords here are the same as in input option
\code{output\_user}. The Legendre polynomials \code{pmom} are only
caclulated if the option \code{nmom} is specified.

If \code{output\_user netcdf} is specified the output is written to a
netcdf file in the format that is required by \code{uvspec}.

\section{Examples}

\subsection{Calculation for one particle}

The following example shows a Mie calculation for a single spherical
particel with a radius of 200~$\mu$m. The refractive index is
specified by the user. The calculation is performed for wavelengths
from 280 to 5000~nm in 5~nm steps.
\efile{../examples/MIE_1.INP}

\subsection{Calculation for a size distribution}

Not all cloud droplets are of one specific size. The cloud droplet
size spectrum may be represented for instance by a gamma
distribution. Gamma distributions can easily be specified using the
option \codeidx{distribution} \code{gamma} as
demonstrated in the following example:
\efile{../examples/MIE_2.INP}
The refractive index of water is taken for this calculation. In order
to generate input for \code{uvspec} Legendre polynomials and from
those the phase matrices need to be calculated. The option
\codeidx{nmom} specifies how many Legendre polynomials shall be
computed. If the selected number is too small for an accurate
representation of the phase matrix, a warning is given. If
\code{output\_user netcdf} is specified the corresponding phase
matrices are calculated from the Legendre moments. The scattering
angle grid is optimized so that the phase matrix is sampled as
accurate as possible. The option \code{nthetamax} can be used to set
an upper limit of scattering angle grid points to be used. 
This example generates a netcdf file which can directly be used in
\code{uvspec} with the options \codeidx{wc\_properties} or
\codeidx{ic\_properties}.


%%% Local Variables: 
%%% mode: latex
%%% TeX-master: t
%%% End: 
